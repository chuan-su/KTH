\documentclass[12pt]{article}

\usepackage[margin=1in]{geometry}
\usepackage{amsmath,amsthm,amssymb}

\newcommand{\N}{\mathbb{N}}
\newcommand{\Z}{\mathbb{Z}}

\newenvironment{theorem}[2][Theorem]{\begin{trivlist}
\item[\hskip \labelsep {\bfseries #1}\hskip \labelsep {\bfseries #2.}]}{\end{trivlist}}
\newenvironment{lemma}[2][Lemma]{\begin{trivlist}
\item[\hskip \labelsep {\bfseries #1}\hskip \labelsep {\bfseries #2.}]}{\end{trivlist}}
\newenvironment{exercise}[2][Exercise]{\begin{trivlist}
\item[\hskip \labelsep {\bfseries #1}\hskip \labelsep {\bfseries #2.}]}{\end{trivlist}}
\newenvironment{problem}[2][Problem]{\begin{trivlist}
\item[\hskip \labelsep {\bfseries #1}\hskip \labelsep {\bfseries #2.}]}{\end{trivlist}}
\newenvironment{question}[2][Question]{\begin{trivlist}
\item[\hskip \labelsep {\bfseries #1}\hskip \labelsep {\bfseries #2.}]}{\end{trivlist}}
\newenvironment{corollary}[2][Corollary]{\begin{trivlist}
\item[\hskip \labelsep {\bfseries #1}\hskip \labelsep {\bfseries #2.}]}{\end{trivlist}}

\newenvironment{solution}{\begin{proof}[Solution]}{\end{proof}}

\begin{document}

\title{Review Questions 3}
\author{Harald Ng \\
        Chuan Su}

\maketitle
\begin{enumerate}
    \item \texttt Both \texttt{DataFrame} and \texttt{DataSet} are distributed collections of data in Spark. \texttt{DataFrame} organized the data into columns and the elements are Rows that itself only are generic untyped JVM objects. This implies that there are no type checking at compile-time and we could get runtime exceptions. The \texttt{DataSet} API unifies the \texttt{DataFrame} and RDD APIs to provide type safety. An example where \texttt{DataSet} is more beneficial is that if the programmer would write the wrong name of a column. \texttt{DataSet} would have a compilation error, but \texttt{DataFrame} would only throw a runtime error.

    \item
       \texttt{Window.rowsBetween(-1, 1)} specifies a frame with \texttt{1 PRECEDING} as the start boundary and \texttt{1 FOLLOWING} as the end boundary, relative to the physical offset from the position of the current input row  \texttt{CURRENT ROW}.

    For every current input row, spark sql calculates the average of people age over the rows that fall into the frame defined by start and end boundary.

    The result is shown as below:
    \begin{verbatim}
    +-------+---+------------------+
    |   name|age|           avg_age|
    +-------+---+------------------+
    |Michael| 15|              22.5|
    |   Andy| 30|21.333333333333332|
    | Justin| 19|20.333333333333332|
    |   Andy| 12|16.666666666666668|
    |    Jim| 19|14.333333333333334|
    |   Andy| 12|              15.5|
    +-------+---+------------------+
    \end{verbatim}

    \item The main difference is that log-based broker systems persist all events in a sequential log. Other message brokers deletes a message once it has been consumed.

    \item For \texttt{Windowing By Processing Time}, stream processing system buffers up incoming data into windows until some amount of processing time has passed, E.g. five-minute fixed window. For \texttt{Windowing By Event Time}, stream processing system buffers up incoming data into windows based on the times at which event actually occured, which is capable of handling out-of-oder events.

    Watermarks flow as part of the data stream and carry a timestamp \texitt{t}, which are used by stream processing system to measure progress in event time and and deal with lateness.
    A watermark \texttt{w(t)} declares that event time has reached time \texttt{t} in the stream, which means there should be no more elements from the stream with a timestamp less than \texttt{t}. If an arriving event lies within the watermark, it gets used to update a query. Otherwise it will start getting dropped.

    \item There are mainly two types of delivery guarantees; \textbf{At-least-once} and \textbf{Exactly-once}. At-least-once guarantees that a message is delivered to the consumer at least once, but could also redeliver the same message multiple times. Exactly-once implies that a message will only be delivered to the consumer exactly once.


\end{enumerate}


\end{document}

\documentclass[12pt]{article}

\usepackage[margin=1in]{geometry}
\usepackage{amsmath,amsthm,amssymb}
\usepackage{listings}% http://ctan.org/pkg/listings
\usepackage{alltt}


\newcommand{\N}{\mathbb{N}}
\newcommand{\Z}{\mathbb{Z}}

\newenvironment{theorem}[2][Theorem]{\begin{trivlist}
\item[\hskip \labelsep {\bfseries #1}\hskip \labelsep {\bfseries #2.}]}{\end{trivlist}}
\newenvironment{lemma}[2][Lemma]{\begin{trivlist}
\item[\hskip \labelsep {\bfseries #1}\hskip \labelsep {\bfseries #2.}]}{\end{trivlist}}
\newenvironment{exercise}[2][Exercise]{\begin{trivlist}
\item[\hskip \labelsep {\bfseries #1}\hskip \labelsep {\bfseries #2.}]}{\end{trivlist}}
\newenvironment{problem}[2][Problem]{\begin{trivlist}
\item[\hskip \labelsep {\bfseries #1}\hskip \labelsep {\bfseries #2.}]}{\end{trivlist}}
\newenvironment{question}[2][Question]{\begin{trivlist}
\item[\hskip \labelsep {\bfseries #1}\hskip \labelsep {\bfseries #2.}]}{\end{trivlist}}
\newenvironment{corollary}[2][Corollary]{\begin{trivlist}
\item[\hskip \labelsep {\bfseries #1}\hskip \labelsep {\bfseries #2.}]}{\end{trivlist}}

\newenvironment{solution}{\begin{proof}[Solution]}{\end{proof}}

\begin{document}

\title{Review Questions 6}
\author{Harald Ng \\
        Chuan Su}

\maketitle
\begin{enumerate}
    \item 
   	The reason is to improve work balance and ensure that the edges are uniformly distributed over machines.
    \item Total resources: \(\langle28CPU, 56GB\rangle\) \\
	User1 wants: \(\langle1CPU, 2GB\rangle\); Dominant resource: Both (1/28 = 2/56) \\
	User2 wants: \(\langle1CPU, 4GB\rangle\); Dominant resource: RAM (1/28 \textless 4/56) \\
	Using dominant Resource Fairness, we should give every user an equal share of its dominant resource.
	\begin{alltt}
	    Maximize Allocation:    max(x,y)        
    CPU Constraint:         x+y \leq\ 28        
    Memory Constraint:      2x + 4y \leq 56 
    Equalize dominant shares:
    \dfrac{x}{28} = \dfrac{4y}{56} \leftrightarrow \dfrac{x}{28} = \dfrac{y}{14} \leftrightarrow x = 2y
	\end{alltt}
	 Put x = 2y into the two other equations and we will get \underline{x = 14} and \underline{y = 7}. This corresponds to User1 getting \(\langle50\% CPU, 50\%GB\rangle\) and User2 getting \(\langle25\% CPU, 50\%GB\rangle\). As we can see, both users get an equal share of the RAM (dominant resource for both user1 and user2).
	 \\
	 Using Asset fairness, we should give weights to resources. In this case we can see that 1 CPU = 2GB. This yields:
	 	\begin{alltt}
	    Maximize Allocation:    max(x,y)        
    CPU Constraint:         x+y \leq\ 28        
    Memory Constraint:      2x + 4y \leq 56 
    Asset fairness:         User1: \(\langle1CPU, 2GB\rangle\) = 1CPU + 1CPU = 2x
                            User2: \(\langle1CPU, 4GB\rangle\) = 1CPU + 2CPU = 3y
                            \rightarrow 2x = 3y
	\end{alltt}
	Which gives a solution of \underline{x = 12} and \underline{y = 8}. This corresponds to User1 getting \(\langle43\% CPU, 43\%GB\rangle\) and User2 getting \(\langle29\% CPU, 57\%GB\rangle\). As we can see, both users get a sum of 86\% resources.
	
	






\end{enumerate}


\end{document}
